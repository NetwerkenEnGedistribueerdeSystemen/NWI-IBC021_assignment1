\documentclass[12pt, a4paper]{article}

\usepackage{enumerate}
\usepackage{multicol}

\setlength\parskip{1em}
\setlength\parindent{0em}

\title{Assignment 1}

\author{Hendrik Werner s4549775}

\begin{document}
\maketitle

\section{} %1
\begin{enumerate}[a]
	\item %a
	\begin{multicols}{2}
		I found 2 network interfaces. One is the localhost (lo) and one is an ethernet interface (eth0).

		\begin{tabular}{c|c}
			Network Interface & IP4\\\hline
			eth0 & 10.0.2.15\\
			lo & 127.0.0.1\\
		\end{tabular}
	\end{multicols}

	\item %b
	\paragraph{ping}
	I began by capturing the traffic on eth0 and issued the command "ping ru.nl". I stopped the process after 4 pings.

	This resulted in a broad casted ARP request to ask who has IP 10.0.2.3. After this request was answered a DNS lookup for ru.nl was issued and the answer contained ru.nl's IP4 address to which a ICMP ping was sent and an answer received. This step was repeated until the process was stopped.

	Interleaved with that was another ARP request to ask who 10.0.2.2 was but I don't know why. This address wasn't used after that.

	I expected the DNS requests to be cached but they were not. For each new ping there was another DNS lookup. The rest is pretty obvious: A request is sent to the looked up IP and an answer received. Then this is repeated until the user stops the process.
\end{enumerate}

\section{} %2
\begin{enumerate}[a]
	\item %a
	\item %b
\end{enumerate}

\section{} %3
\begin{enumerate}[a]
	\item %a
	\item %b
	\item %c
\end{enumerate}

\section{} %4
\begin{enumerate}[a]
	\item %a
	\item %b
	\item %c
\end{enumerate}

\section{} %5
\begin{enumerate}[a]
	\item %a
	\item %b
	\item %c
\end{enumerate}

\end{document}
